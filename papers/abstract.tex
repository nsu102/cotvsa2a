체인 오브 생각(Chain-of-Thought,
  CoT) 프롬프팅은 대규모 언어 모델의
  추론 능력을 향상시키지만, 긴 추론
  과정으로 인해 토큰 사용량이 크게
  증가한다는 문제가 있다. 본
  연구에서는 에이전트 간
  통신(Agent-to-Agent, A2A)
  프로토콜을 활용한 토큰 효율적 추론
  방법을 제안한다. 제안된 방법은
  Planner와 Solver 두 개의 전문화된
  에이전트로 문제 분석과 해결을
  분리하여, CoT와 유사한 정확도를
  달성하면서도 토큰 사용량을 약 50-60% 
  절감한다. MATH-500 데이터셋을 이용한
   실험에서 Claude Haiku 4.5와
  GPT-4o-mini 모델에 대해 제안된 A2A
  방법이 전통적인 단일 프롬프트
  방식보다 토큰 효율성이 우수함을
  확인하였다.